\documentclass{article}
\usepackage{mdframed}
\input{preamble}

\pagestyle{fancy}
\fancyhf{}

\fancyhead[L]{MATH463 Notes}
\fancyhead[R]{Complex Variables}

\fancyfoot[L]{Mazin Karjikar}
\fancyfoot[R]{\thepage}

\newcommand{\framethis}[2][1.3]{
  \begin{framed}
    \begin{spacing}{#1}
      #2
    \end{spacing}
  \end{framed}
}


\begin{document}

  \begin{titlepage}
    \centering
    \vspace*{3cm}
    
    \Huge \textbf{Complex Variables} \par
    \Huge \textbf{MATH463} \par
    \vspace{1cm}
    \Huge Mazin Karjikar\par
    \vspace{1cm}
    \huge Spring 2024 \par
    \vspace{2cm}
    \Large
    Lecture notes for Amin Gholampour's class taught 
    at the University of Maryland.
    
  \end{titlepage}

  \tableofcontents

  \newpage

  \section{Review of Complex Numbers} \vspace{7mm}

  \definition{A \textbf{complex number} $z = x + iy \in \C$ 
  consists of a \textbf{real part} $\re z = x \in \R $ and an \textbf{imaginary part}
  $\im z = y \in \R$. } 
  % \vspace{5mm}

  \begin{framed}\begin{spacing}{1.2}
    
    $\C$ (the set of complex numbers) is equipped with \textbf{addition}
    and \textbf{multiplication}. 

  Let $z_1 = x_1 + iy_1$, $z_2 = x_z + iy_2 \in \C$. Then \el
  $z_1 + z_2 = \parens{x_1+x_2} + i\parens{y_1+y_2}$ and \el
  $z_1 \cdot z_2 = \parens{x_1 \cdot x_2 - y_1 \cdot y_2} + 
  i\parens{x_1 \cdot y_2 + y_1 \cdot x_2}$. 
  \end{spacing}
  \end{framed} 
  % \vspace{5mm}

  \example{Try adding and multiplying two complex numbers. Addition has a simple geometric interpretation.}
  % \vspace{5mm}
  
  \definition{A \textbf{field} is a set $F$ equipped with and closed under two binary operators. The operators must be 
  \textbf{associative}, \textbf{commutative}, and \textbf{distributive}. Each operator must also have
  an \textbf{inverse} and an \textbf{identity}.}
  % \vspace{5mm}

  \begin{framed}\begin{spacing}{1.4}
    $\parens{\C, +, \cdot}$ forms a \textbf{field}. For any $z_1, z_2, z_3
    \in \C$, \\
    $(z_1+z_2)+z_3 = z_1+(z_2+z_3)$ and $(z_1z_2)z_3 = z_1(z_2z_3) \rightarrow$ \textbf{Associativity}. \\
    $z_1+z_2 = z_2+z_1$ and $z_1z_2 = z_2z_1$ $\rightarrow$ \textbf{Commutativity}. \\
    $z_1(z_2+z_3) = z_1z_2 + z_1z_3 \rightarrow$ \textbf{Distributivity}. \\
    $0 = 0 + 0i \rightarrow$ \textbf{Additive Identity}. \\
    $1 = 1 + 0i \rightarrow$ \textbf{Multiplicative Identity}. \\
    $x+iy \mapsto -x-iy \rightarrow$ \textbf{Additive Inverse}. \\
    $\dis 0 \neq x+iy \mapsto \frac{x}{x^2+y^2} -i\frac{y}{x^2+y^2} \rightarrow$ \textbf{Multiplicative Inverse}.
  \end{spacing}
  \end{framed}

  \definition{The \textbf{absolute value} (or \textbf{modulus}) of a complex number $z =x+iy$ is denoted by $\abs{z} = \sqrt{x^2 + y^2}$.
  It is a nonnegative real number. It is gemetrically similar to the magnitude.}

  \definition{The \textbf{complex conjugate} of $z=x+iy$ is $\overline{z} = x-iy$.}
  
  \framethis{\textbf{Geometric Interpretation}: $\abs{z}$ is the distance from 
  $z$ to 0. $\overline{z}$ is obtained by reflecting $z$ over the real axis.}

  \framethis[]{\textbf{Basic Properties:} For $z_1,z_2 \in \C$, $a \in \R > 0$,\el
  $\dis \overline{(z_1+z_2)} = \overline{z_1}+\overline{z_2}$ and $\overline{z_1z_2} = \overline{z_1} \cdot \overline{z_2}$,\el
  $\abs{z_1z_2} = \abs{z_2z_1}$ and $\abs{az_1} = a\abs{z_1}$,\el
  $\abs{z_1+z_2} \leq \abs{z_1}+\abs{z_2} \rightarrow$ \textbf{Triangle Inequality}. Equality occurs when $0$, $z_1$, and $z_2$ are collinear.\el
  $z_1\overline{z_1} = \abs{z_1}^2$.\el
  By the last property, if $z_1 \neq 0$, $\dis z_1^{-1} = \frac{1}{z_1} = \frac{\overline{z_1}}{z_1\overline{z_1}} = \frac{\overline{z_1}}{\abs{z_1}^2}$.}

  \example{$\dis i^{-1} = \frac{1}{i} = \frac{-i}{1} = -i$. $\dis (4 + 3i)^{-1} = \frac{(4-3i)}{25}$.}

  \framethis{\textbf{Polar Form of Complex Numbers:}\el
  The \textbf{polar form} of a complex number $z = x+iy \coloneqq \abs{z} (\cos\theta + i \sin\theta)$.
  This is basically the magnitude/modulus of the complex number multiplied by the direction it faces.
  Thus we can let $r = \abs{z}$ and $e^{i\theta} = \cos\theta + i\sin\theta$, and we get
  $z = re^{i\theta}$. Notice if $z \neq 0$, then $\dis \frac{z}{\abs{z}}$ lies on the unit circle.
  }

  \definition{$\theta$ is called the \textbf{argument} of complex number $z \neq 0$. If $\theta$
  is an argument of $z$, then so is $\theta + 2n\pi$ for any $n \in \Z$.\el
  The \textbf{principal argument} of $z \neq 0$ denoted by $\Arg z$ is the unique $-\pi < \theta \leq \pi$ that
  satisfies $ z = \abs{z}e^{i\theta}$.}

  \framethis{\textbf{Some Polar Properties:}\el
    If $z_1 = r_1e^{i\theta_1}$, $z_2 = r_2e^{i\theta_2}$, then $z_1z_2=r_1r_2e^{i(\theta_1+\theta_2)}$.
    Geometric interpretation: To multiply two complex numbers, multiply their moduli and add their arguments.\el
    If $z = re^{i\theta}$, then for any $n \in \Z$, $z^n=r^n e^{in\theta}$. Note that for negative $n$, $z^{-1} = 
    \dis \frac{1}{r}e^{-i\theta}.$\el
    If $z_1 = r_1e^{i\theta_1}$, $z_2 = r_2e^{i\theta_2}$, then $\dis\frac{z_1}{z_2}
    = \frac{r_1}{r_2}e^{i(\theta_1 - \theta_2)}$.   
  }

  \example{$(1+i)^{18} \mapsto \sqrt{2}^{18}e^{i18\pi/4}$ in polar. Then we want the argument
  to be the principal argument. $ = 2^9e^{i\pi/2}$. Now we can see $= 2^9i$.}

  \framethis{\textbf{Roots of Complex Numbers:} Any nonzero complex number has exactly
  $n$ $n^{th}$ roots.\el
  IF $z = re^{i\theta}$, then the distinct $n^{th}$ roots of $z$ are 
  $\dis r^{\frac{1}{n}}e^{i(\frac{\theta}{n} + \frac{2k\pi}{n})}$ for 
  $k = 0,1, \ldots n-1$.\el
  \textbf{Geometric Interpretation:} The $n^{th}$ roots of $z$ are located on the
  circle with center 0 and radius $\abs{z}^{\frac{1}{n}}$ and they divide the circle
  into $n$ equal parts.}

  \framethis{\textbf{Solutions to Quadratics:} Given a quadratic polynomial 
  $az^2+bz+c$, where $a,b,c \in \C$, define $\Delta = b^2-4ac$. If $\Delta \neq 0$,
  it always has two square roots $\sqrt{\Delta}$ and $-\sqrt{\Delta}$. So the roots
  of the quadratic are $\dis \frac{-b\pm \sqrt{\Delta}}{2a}$.}
\end{document}
