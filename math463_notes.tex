\documentclass{article}
\usepackage{mdframed}
\usepackage{amssymb}
\usepackage{amsmath, amsthm}
\usepackage{xcolor}
\usepackage{fancyhdr}
\usepackage{enumitem}
\usepackage{mathtools}
\usepackage{framed}
\usepackage{parskip}
\usepackage{graphicx}
\usepackage{chngcntr}
\usepackage{float}
\usepackage{listings}
\usepackage{inconsolata}
\usepackage{transparent}
\usepackage{tikz}
\usepackage{hyperref}
\usepackage{setspace}
\usepackage[margin=1in, top=1in, bottom=1in]{geometry}

\def\A{{\mathbb A}}
\def\P{{\mathbb P}}
\def\N{{\mathbb N}}
\def\Z{{\mathbb Z}}
\def\Q{{\mathbb Q}}
\def\R{{\mathbb R}}
\def\C{{\mathbb C}}
\def\F{{\mathbb F}}
\def\O{{\cal O}}
\let\sec\S
\let\S\relax
\def\S{{\mathfrak S}}
\def\g{{\mathfrak g}}
\def\p{{\mathfrak p}}
\def\h{{\mathfrak h}}
\def\n{{\mathfrak n}}
\def\v{{\mathfrak v}}
\def\m{{\mathfrak m}}
\def\a{{\alpha}}

\newcommand{\skipline}{\vspace{\baselineskip}}
\newcommand{\dis}{\displaystyle}
\newcommand{\noin}{\noindent}
\newcommand{\el}{\vspace{-3mm} \par}

% remove all paragraph indents
\setlength{\parindent}{0pt}
\setlength{\parskip}{5mm}

% Figure counter include section
\counterwithin{figure}{section}

% Cleaner figures
\newcommand{\fig}[3][0.4]{
  \begin{figure}[H]
    \centering
    \includegraphics[width=#1\textwidth, keepaspectratio]{#2}
    \caption{#3}
  \end{figure}
}

% Parens, Brackets, Bars, and Braces
\newcommand{\parens}[1]{ \left(#1\right) }
\newcommand{\bracks}[1]{ \left[#1\right] }
\newcommand{\braces}[1]{ \left\{#1\right\} }
\newcommand{\abs}[1]{ \left|#1\right| }
\newcommand{\floor}[1]{ \left\lfloor#1\right\rfloor }
\newcommand{\ceil}[1]{ \left\lceil#1\right\rceil }

% Mathematical notation


\newcommand{\Span}{\mathrm{Span}}
\newcommand{\Range}{\mathrm{Range}}
\newcommand{\Null}{\mathrm{Null}}
\newcommand{\Rank}{\mathrm{Rank}}
\newcommand{\rank}{\mathrm{rank}}
\newcommand{\Nullity}{\mathrm{Nullity}}
\newcommand{\nullity}{\mathrm{nullity}}
\newcommand{\longhookrightarrow}{\lhook\joinrel\relbar\joinrel\rightarrow}
\newcommand{\la}{\leftarrow}
\newcommand{\ra}{\rightarrow}
\newcommand{\La}{\Leftarrow}
\newcommand{\Ra}{\Rightarrow}
\newcommand{\dbar}{\overline{\partial}}
\newcommand{\gequ}{\geqslant}
\newcommand{\lequ}{\leqslant}
\newcommand{\Hom}{\mathrm{Hom}}
\newcommand{\End}{\mathrm{End}}
\newcommand{\Aut}{\mathrm{Aut}}
\newcommand{\Coker}{\mathrm{Coker}}
\newcommand{\Row}{\mathrm{Row}}
\newcommand{\Ker}{\mathrm{Ker}}
\newcommand{\Tr}{\mathrm{Tr}}
\newcommand{\Id}{\mathrm{Id}}
% \newcommand{\mod}{\mathrm{mod }}
\newcommand{\un}{\underline}
\newcommand{\ov}{\overline}
\newcommand{\wt}{\widetilde}
\newcommand{\wh}{\widehat}
\newcommand{\pr}{\prime}
\newcommand{\rk}{\mathrm{rk}}
\newcommand{\im}{\mathrm{Im}\,}
\newcommand{\re}{\mathrm{Re}\,}
\newcommand{\Arg}{\mathrm{Arg}\,}

% Linear Algebra

\newcommand{\lind}{linearly independent}
\newcommand{\ldep}{linearly dependent}
\renewcommand{\vec}[1]{
  {\bf #1}
}
\newcommand{\lincomb}[3]{
  #1_{1}#2_{1} + #1_{2}#2_{2} + \cdots + #1_{#3}#2_{#3}
}
\newcommand{\neglincomb}[3]{
  -#1_{1}#2_{1} - #1_{2}#2_{2} - \cdots - #1_{#3}#2_{#3}
}
\newcommand{\constants}[2]{
  #1_{1}, #1_{2}, \cdots, #1_{#2}
}
\newcommand{\constantsz}[2]{
  #1_{0}, \constants{#1}{#2}
}

% ================= %
% Headers & Footers
% ================= %
\pagestyle{fancy}
\fancyhf{}
\newcommand{\intros}[3]{
  \lhead{\textbf{#1} {#2}}
  \rhead{#3}}
\rfoot{\thepage}
\renewcommand{\headrulewidth}{0pt}



% ================= %
%       Utils
% ================= %
\newcommand{\induction}[3]{
  \textbf{Base Case}. #1 \\
  \textbf{Inductive Hypothesis}. \\ #2 \\
  \textbf{Inductive Step}. \\ #3
}



% Used to list all problems on homework
\newcommand{\problems}[1]{
  \medskip \noin
  {\bf Problems}

  #1

  \medskip{}
}


% When prof does Question/Answer styling
\newcommand{\qna}[2]{
  {\bf Question}: #1

  {\bf Answer}: #2
}


\newcommand{\littletaller}{\mathchoice{\vphantom{\big|}}{}{}{}}

% ================= %
%      Box Meta
% ================= %

% #2 - FG Color
% #3 - BG Color
\newenvironment{fancyleftbar}[3][\hsize]
{%
    \def\FrameCommand
    {%
        {\color{#2}\vrule width 3pt}%
        \hspace{0pt}%must no space.
        \fboxsep=\FrameSep\colorbox{#3}%
    }%
    \MakeFramed{\hsize#1\advance\hsize-\width\FrameRestore}%
}
{\endMakeFramed}

\newenvironment{simpleleftbar}[3][\hsize]
{%
    \def\FrameCommand
    {%
        {\vrule width 0.5pt}%
        \hspace{3pt}%
        \fboxsep=\FrameSep%
        \colorbox{#3}%
    }%
    \MakeFramed{\hsize#1\advance\hsize-\width\FrameRestore}%
}
{\endMakeFramed}

% Used to allow the color argument to pass through the environment%
\newsavebox{\boxqed} 

% #1 - Header
% #2 - FG Color
% #3 - BG Color
\newenvironment{fancybox}[3]{
  \sbox\boxqed{\textcolor{#2}{$\blacksquare$}}
  \begin{fancyleftbar}{#2}{#3}

  \noin
  #1
  % {\large \bf \underline{#1}}
  \smallskip\noin \\
}
{

  \medskip
  \noin
  \usebox\boxqed

  \end{fancyleftbar}
}

% #1 - Text header
% #2 - Outer Text
% #3 - Inner Text
% #4 - Inner Header
% #5 - FG Color
% #6 - Background Color
\newcommand{\boxmeta}[6]{
  #1
  % {\small\sc\uppercase{#1}}

  #2

  \begin{fancybox}{#4}{#5}{#6}
    \noin
    #3
  \end{fancybox}
}

% #1 - Title
% #2 - FG Color
% #3 - BG Color
% #4 - Inner Text
\newcommand{\baronly}[4]{
  \begin{simpleleftbar}{#2}{#3}
    {\bf #1}.
    \vspace{-3mm}
    \begin{spacing}{1.3}
      #4
    \end{spacing}
  \end{simpleleftbar}
}

% ================= %
%     Box Colors
% ================= %

\definecolor{theorem_fg}{HTML}{EABAC3}
\definecolor{theorem_bg}{HTML}{F9EEF0}

\definecolor{problem_fg}{HTML}{ABABAB}
\definecolor{problem_bg}{HTML}{EDEDED}

\definecolor{lemma_fg}{HTML}{D0C97D}
\definecolor{lemma_bg}{HTML}{FCF9DB}

\definecolor{prop_fg}{HTML}{7DDB89}
\definecolor{prop_bg}{HTML}{D7FADB}

\definecolor{defn_fg}{HTML}{83D4CF}
\definecolor{defn_bg}{HTML}{E7FCFB}

\definecolor{lst_bg}{HTML}{EFF6F8}
\definecolor{lst_fg}{HTML}{475857}

\definecolor{btw_fg}{HTML}{5A5A5A}

% ================= %
%     Box Envs
% ================= %

\newcommand{\Definition}[2]{
                              \boxmeta{}{}{#2}{{\it Definition}. {\bf\underline{#1}}}{defn_fg}{defn_bg}
                            }

\newcommand{\Theorem}[2]{
  \boxmeta{{\bf Theorem.}}{#1}{#2}{{\bf Proof.}}{theorem_fg}{theorem_bg}
}

\newcommand{\NamedTheorem}[3]{
  \boxmeta{#1}{#2}{#3}{Proof}{theorem_fg}{theorem_bg}
}

\newcommand{\Problem}[3]{
  \boxmeta{Problem #1}{#2}{#3}{Solution}{problem_fg}{problem_bg}
}

\newcommand{\Example}[2]{
  \boxmeta{Example}{#1}{#2}{}{problem_fg}{problem_bg}
}

\newcommand{\Lemma}[2]{
  \boxmeta{{\bf Lemma.}}{#1}{#2}{{\bf Proof.}}{lemma_fg}{lemma_bg}
}

\newcommand{\NamedLemma}[3]{
  \boxmeta{#1}{#2}{#3}{Proof}{lemma_fg}{lemma_bg}
}

\newcommand{\Corollary}[2]{
  \boxmeta{Corollary}{#1}{#2}{Proof}{lemma_fg}{lemma_bg}
}

\newcommand{\Proposition}[2]{
  \boxmeta{Proposition}{#1}{#2}{Proof}{prop_fg}{prop_bg}
}

% ================== %
%      Bar Only
% ================== %

\newcommand{\definition}[1]{
  \baronly{Definition}{defn_fg}{defn_bg}{#1}
}

\newcommand{\theorem}[1]{
  \baronly{Theorem}{theorem_fg}{theorem_bg}{#1}
}

\newcommand{\example}[1]{
  \baronly{Example}{problem_fg}{problem_bg}{#1}
}

\newcommand{\remark}[1]{
  \baronly{Remark}{problem_fg}{problem_bg}{#1}
}

\newcommand{\note}[1]{
  \medskip
  \baronly{Note}{problem_fg}{problem_bg}{#1}
}

\newcommand{\btw}[1]{
  {\bf Author Note}.

  \begin{color}{btw_fg}
    #1
  \end{color}
}

\newcommand{\sidenote}[1]{
  {\bf Side Note}.

  \boxmeta{}{}{#1}{}{problem_fg}{problem_bg}
}



% graphs
\def\deg{\text{deg}}
\def\indeg{\text{indeg}}
\def\outdeg{\text{outdeg}}

% big O notation
\def\O{\mathcal O}

% define parent
\def\pr{\text{pr}}

% define incomplete commands
\def\TODO{\color{red}\textbf{TODO}\color{black}\,}
\def\QUESTION{\color{red}\textbf{QUESTION}\color{black}\,}



% listings settings
\lstset{
  % general styles
  backgroundcolor=\color{lst_bg},
  numbers=left,
  numberstyle=\color{lst_fg}\ttfamily\textbf,
  numbersep=3mm,
  frame=l,
  framesep=7mm,
  framexleftmargin=1.5mm,
  fillcolor=\color{lst_bg},
  rulecolor=\color{lst_bg},
  xleftmargin=9mm,
  % keyword styles
  keywordstyle=[1]\textbf,
  keywordstyle=[2]\textit,
  keywordstyle=[3]\textbf\textit,
  keywords=[1]{let, for, while, not, if, else, then, do, end, return},
  keywords=[2]{if, condition},
  keywords=[3]{do},
  mathescape=true, % enable math mode in listings
  columns=fullflexible,
  basicstyle=\ttfamily % this font looks a bit better than the default
}

\pagestyle{fancy}
\fancyhf{}

\fancyhead[L]{MATH463 Notes}
\fancyhead[R]{Complex Variables}

\fancyfoot[L]{Mazin Karjikar}
\fancyfoot[R]{\thepage}

\newcommand{\framethis}[2][1.3]{
  \begin{framed}
    \begin{spacing}{#1}
      #2
    \end{spacing}
  \end{framed}
}


\begin{document}

  \begin{titlepage}
    \centering
    \vspace*{3cm}

    \Huge \textbf{Complex Variables} \par
    \Huge \textbf{MATH463} \par
    \vspace{1cm}
    \Huge Mazin Karjikar\par
    \vspace{1cm}
    \huge Spring 2024 \par
    \vspace{2cm}
    \Large
    Lecture notes for Amin Gholampour's class taught 
    at the University of Maryland.
    
  \end{titlepage}

  \tableofcontents

  \newpage

  \section{Review of Complex Numbers} \vspace{7mm}

  \definition{A \textbf{complex number} $z = x + iy \in \C$ 
  consists of a \textbf{real part} $\re z = x \in \R $ and an \textbf{imaginary part}
  $\im z = y \in \R$. } 
  % \vspace{5mm}

  \begin{framed}\begin{spacing}{1.2}
    
    $\C$ (the set of complex numbers) is equipped with \textbf{addition}
    and \textbf{multiplication}. 

  Let $z_1 = x_1 + iy_1$, $z_2 = x_z + iy_2 \in \C$. Then \el
  $z_1 + z_2 = \parens{x_1+x_2} + i\parens{y_1+y_2}$ and \el
  $z_1 \cdot z_2 = \parens{x_1 \cdot x_2 - y_1 \cdot y_2} + 
  i\parens{x_1 \cdot y_2 + y_1 \cdot x_2}$. 
  \end{spacing}
  \end{framed} 
  % \vspace{5mm}

  \example{Try adding and multiplying two complex numbers. Addition has a simple geometric interpretation.}
  % \vspace{5mm}
  
  \definition{A \textbf{field} is a set $F$ equipped with and closed under two binary operators. The operators must be 
  \textbf{associative}, \textbf{commutative}, and \textbf{distributive}. Each operator must also have
  an \textbf{inverse} and an \textbf{identity}.}
  % \vspace{5mm}

  \begin{framed}\begin{spacing}{1.4}
    $\parens{\C, +, \cdot}$ forms a \textbf{field}. For any $z_1, z_2, z_3
    \in \C$, \\
    $(z_1+z_2)+z_3 = z_1+(z_2+z_3)$ and $(z_1z_2)z_3 = z_1(z_2z_3) \rightarrow$ \textbf{Associativity}. \\
    $z_1+z_2 = z_2+z_1$ and $z_1z_2 = z_2z_1$ $\rightarrow$ \textbf{Commutativity}. \\
    $z_1(z_2+z_3) = z_1z_2 + z_1z_3 \rightarrow$ \textbf{Distributivity}. \\
    $0 = 0 + 0i \rightarrow$ \textbf{Additive Identity}. \\
    $1 = 1 + 0i \rightarrow$ \textbf{Multiplicative Identity}. \\
    $x+iy \mapsto -x-iy \rightarrow$ \textbf{Additive Inverse}. \\
    $\dis 0 \neq x+iy \mapsto \frac{x}{x^2+y^2} -i\frac{y}{x^2+y^2} \rightarrow$ \textbf{Multiplicative Inverse}.
  \end{spacing}
  \end{framed}

  \definition{The \textbf{absolute value} (or \textbf{modulus}) of a complex number $z =x+iy$ is denoted by $\abs{z} = \sqrt{x^2 + y^2}$.
  It is a nonnegative real number. It is gemetrically similar to the magnitude.}

  \definition{The \textbf{complex conjugate} of $z=x+iy$ is $\overline{z} = x-iy$.}
  
  \framethis{\textbf{Geometric Interpretation}: $\abs{z}$ is the distance from 
  $z$ to 0. $\overline{z}$ is obtained by reflecting $z$ over the real axis.}

  \framethis[]{\textbf{Basic Properties:} For $z_1,z_2 \in \C$, $a \in \R > 0$,\el
  $\dis \overline{(z_1+z_2)} = \overline{z_1}+\overline{z_2}$ and $\overline{z_1z_2} = \overline{z_1} \cdot \overline{z_2}$,\el
  $\abs{z_1z_2} = \abs{z_1}\abs{z_2}$ and $\abs{az_1} = a\abs{z_1}$,\el
  $\abs{z_1+z_2} \leq \abs{z_1}+\abs{z_2} \rightarrow$ \textbf{Triangle Inequality}. Equality occurs when $0$, $z_1$, and $z_2$ are collinear.\el
  $z_1\overline{z_1} = \abs{z_1}^2$.\el
  By the last property, if $z_1 \neq 0$, $\dis z_1^{-1} = \frac{1}{z_1} = \frac{\overline{z_1}}{z_1\overline{z_1}} = \frac{\overline{z_1}}{\abs{z_1}^2}$.}

  \example{$\dis i^{-1} = \frac{1}{i} = \frac{-i}{1} = -i$. $\dis (4 + 3i)^{-1} = \frac{(4-3i)}{25}$.}

  \framethis{\textbf{Polar Form of Complex Numbers:}\el
  The \textbf{polar form} of a complex number $z = x+iy \coloneqq \abs{z} (\cos\theta + i \sin\theta)$.
  This is basically the magnitude/modulus of the complex number multiplied by the direction it faces.
  Thus we can let $r = \abs{z}$ and $e^{i\theta} = \cos\theta + i\sin\theta$, and we get
  $z = re^{i\theta}$. Notice if $z \neq 0$, then $\dis \frac{z}{\abs{z}}$ lies on the unit circle.
  }

  \definition{$\theta$ is called the \textbf{argument} of complex number $z \neq 0$. If $\theta$
  is an argument of $z$, then so is $\theta + 2n\pi$ for any $n \in \Z$.\el
  The \textbf{principal argument} of $z \neq 0$ denoted by $\Arg z$ is the unique $-\pi < \theta \leq \pi$ that
  satisfies $ z = \abs{z}e^{i\theta}$.}

  \framethis{\textbf{Some Polar Properties:}\el
    If $z_1 = r_1e^{i\theta_1}$, $z_2 = r_2e^{i\theta_2}$, then $z_1z_2=r_1r_2e^{i(\theta_1+\theta_2)}$.
    Geometric interpretation: To multiply two complex numbers, multiply their moduli and add their arguments.\el
    If $z = re^{i\theta}$, then for any $n \in \Z$, $z^n=r^n e^{in\theta}$. Note that for negative $n$, $z^{-1} = 
    \dis \frac{1}{r}e^{-i\theta}.$\el
    If $z_1 = r_1e^{i\theta_1}$, $z_2 = r_2e^{i\theta_2}$, then $\dis\frac{z_1}{z_2}
    = \frac{r_1}{r_2}e^{i(\theta_1 - \theta_2)}$.   
  }

  \example{$(1+i)^{18} \mapsto \sqrt{2}^{18}e^{i18\pi/4}$ in polar. Then we want the argument
  to be the principal argument. $ = 2^9e^{i\pi/2}$. Now we can see $= 2^9i$.}

  \framethis{\textbf{Roots of Complex Numbers:} Any nonzero complex number has exactly
  $n$ $n^{th}$ roots.\el
  If $z = re^{i\theta}$, then the distinct $n^{th}$ roots of $z$ are 
  $\dis r^{\frac{1}{n}}e^{i(\frac{\theta}{n} + \frac{2k\pi}{n})}$ for 
  $k = 0,1, \ldots n-1$.\el
  \textbf{Geometric Interpretation:} The $n^{th}$ roots of $z$ are located on the
  circle with center 0 and radius $\abs{z}^{\frac{1}{n}}$ and they divide the circle
  into $n$ equal parts.}

  \framethis{\textbf{Solutions to Quadratics:} Given a quadratic polynomial 
  $az^2+bz+c$, where $a,b,c \in \C$, define $\Delta = b^2-4ac$. If $\Delta \neq 0$,
  it always has two square roots $\sqrt{\Delta}$ and $-\sqrt{\Delta}$. So the roots
  of the quadratic are $\dis \frac{-b\pm \sqrt{\Delta}}{2a}$.}
\end{document}
